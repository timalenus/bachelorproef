%%=============================================================================
%% Methodologie
%%=============================================================================

\chapter{\IfLanguageName{dutch}{Methodologie}{Methodology}}
\label{ch:methodologie}

%% TODO: Hoe ben je te werk gegaan? Verdeel je onderzoek in grote fasen, en
%% licht in elke fase toe welke stappen je gevolgd hebt. Verantwoord waarom je
%% op deze manier te werk gegaan bent. Je moet kunnen aantonen dat je de best
%% mogelijke manier toegepast hebt om een antwoord te vinden op de
%% onderzoeksvraag.


%%praktische aanpak: algemeen uitleggen op basis van historische gegevens
\section{Type onderzoek}
Het onderzoek beschreven in deze tekst is een kwantitatief onderzoek. De te ontwikkelen simulatietool beoogt een service level te berekenen voor de verschillende gesimuleerde situaties. Aan de hand van onder andere dit service level kunnen de verschillende situaties vergeleken worden. De situaties zullen onderling verschillen door het aantal reservaties in de dataset, het aantal auto's en de manier van toewijzen van een auto aan een reservatie. Naast het vergelijken van het service level van de verschillende situaties zullen we ook de gebruiksduur per auto voor een gegeven situatie vergelijken. Zo kan het immers zijn dat voor een bepaalde waarde van het service level één van de auto's slechts een zeer kleine gebruiksduur heeft. In zulke situaties is het dan extra interessant om de simulatie opnieuw uit te voeren met 1 auto minder en de impact op het service level te meten. 

%%het eigenlijke maken van de dataset uitleggen

%%rol van de simulatietool uitleggen (maar neit in detail, berekent service level enzo)


