%%=============================================================================
%% Samenvatting
%%=============================================================================

% TODO: De "abstract" of samenvatting is een kernachtige (~ 1 blz. voor een
% thesis) synthese van het document.
%
% Deze aspecten moeten zeker aan bod komen:
% - Context: waarom is dit werk belangrijk?
% - Nood: waarom moest dit onderzocht worden?
% - Taak: wat heb je precies gedaan?
% - Object: wat staat in dit document geschreven?
% - Resultaat: wat was het resultaat?
% - Conclusie: wat is/zijn de belangrijkste conclusie(s)?
% - Perspectief: blijven er nog vragen open die in de toekomst nog kunnen
%    onderzocht worden? Wat is een mogelijk vervolg voor jouw onderzoek?
%
% LET OP! Een samenvatting is GEEN voorwoord!

%%---------- Nederlandse samenvatting -----------------------------------------
%
% TODO: Als je je bachelorproef in het Engels schrijft, moet je eerst een
% Nederlandse samenvatting invoegen. Haal daarvoor onderstaande code uit
% commentaar.
% Wie zijn bachelorproef in het Nederlands schrijft, kan dit negeren, de inhoud
% wordt niet in het document ingevoegd.

\IfLanguageName{english}{%
\selectlanguage{dutch}
\chapter*{Samenvatting}
\selectlanguage{english}
}{}

%%---------- Samenvatting -----------------------------------------------------
% De samenvatting in de hoofdtaal van het document

\chapter*{\IfLanguageName{dutch}{Samenvatting}{Abstract}}
Partago is een elektrische autodeel coöperatie in volle groei. Om elektrisch autodelen zorgeloos te maken ontwikkelde Partago een eigen autodeelplatform. Als gebruikersaantallen van Partago blijven groeien zal er vroeg of laat een opschaling nodig zijn van het huidige reserveringssysteem. Dit onderzoek onderzoekt één van de mogelijke optimalisaties: de gebruiker niet zelf een auto laten kiezen tijdens het reserveren, maar een auto laten toewijzen door een algoritme zodat de beschikbare auto's optimaal worden ingevuld. Zulk toewijzingsprobleem kan geformuleerd worden aan de hand van een Constraint Satisfaction Problem, afgekort tot CSP. Op basis van een dataset van bestaande reservaties werden, met een zelf gecodeerde simulatietool, simulaties uitgevoerd voor een periode van 4 weken telkens met een verschillend aantal reservaties en een verschillend aantal beschikbare auto's. Voor elke simulatie werd de toewijzing van auto's aan reservaties éénmaal willekeurig gedaan en éénmaal aan de hand van het oplossen van een Constraint Satisfaction Problem. Voor zowel de eenvoudige toewijzing als voor de toewijzing met behulp van een CSP werd het service level, gedefinieerd als het percentage van reservaties dat kan doorgaan, en de totale actieve tijd per auto berekend. Uit de resultaten van de simulaties kan afgeleid worden dat er maar kleine winstmarges te boeken zijn voor het service level en de actieve tijd per auto door gebruik te maken van een CSP in vergelijking met de eenvoudige toewijzing. Naarmate het systeem drukker en complexer wordt, meer reservaties en meer auto's, lijken de winstmarges te groeien. Door het gebruik van een niet-performant algoritme om het CSP op te lossen was dit onderzoek gelimiteerd om de complexiteit van de gesimuleerde systemen op te drijven. Een vervolg op dit onderzoek zou een meer geavanceerd algoritme kunnen gebruiken om het CSP op te lossen.
