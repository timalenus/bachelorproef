\chapter{\IfLanguageName{dutch}{Stand van zaken}{State of the art}}
\label{ch:stand-van-zaken}

% Tip: Begin elk hoofdstuk met een paragraaf inleiding die beschrijft hoe
% dit hoofdstuk past binnen het geheel van de bachelorproef. Geef in het
% bijzonder aan wat de link is met het vorige en volgende hoofdstuk.

% Pas na deze inleidende paragraaf komt de eerste sectiehoofding.

\section{Wat is autodelen?}
Het fenomeen autodelen bestaat ondertussen zo'n 15-tal jaar in België \autocite{ing}, autodelen is echter geen recent fenomeen. Volgens \textcite{millardball} vinden we de eerste pogingen van een vastgestelde autodeelorganisatie terug in de jaren '40 in een coöperatief samenwoon-project in het Zwitserse Zürich. Het meeste onderzoek gedaan rond autodelen geeft echter geen formele definitie van wat autodelen nu exact is \autocite{millardball}. In dit onderzoek zullen we autodelen definiëren als "de toegang, via een abonnement of bundel, tot een vloot auto's via het zelfbedieningsprincipe, beschikbaar voor korte periodes en korte afstanden, waarvoor de gebruiker (particulier of bedrijf) een toeslag per uur en/of per afgelegde kilometer betaalt" \autocite{ing}.

\section{Wachtrijtheorie}
In de masterthesis "Dimensionering van een autodeelsysteem aan de hand van wachtrijtheorie" beschrijft \textcite{van-buggenhout} hoe een reservatiesysteem zoals geïmplementeerd door Partago gemodeleerd kan worden aan de hand van wachtrijtheorie. De bevindingen van Van Buggenhout zijn uitermate interessant voor dit onderzoek. In haar werk wordt een grote voorzet gegeven hoe we voor de beoogde simulatietool van dit onderzoek een wiskundig model kunnen creëren. In wat volgt vatten we kort samen hoe het systeem gemodelleerd kan worden aan de hand van wachtrijtheorie.

\subsection{Wat is wachtrij theorie}
Een wachtrijmodel beschrijft een systeem waarbij klanten arriveren bij het systeem op verschillende tijdstippen, wachten tot het hun beurt is om door het systeem geserveerd te worden en tenslotte het systeem terug verlaten. Het doel van wachtrijtheorie is een balans te vinden tussen de service naar klanten toe en het aantal servers. Meer servers betekend doorgaans hogere kosten. \textcite{van-buggenhout}. Naar Partago toe kunnen we de servers vertalen naar auto's. Hoe meer auto's hoe meer klanten Partago kan serveren, maar hoe meer auto's hoe hoger de operationele kosten. 

Om het systeem correct te modelleren aan de hand van wachtrijtheorie moeten we de volgende parameters kennen:
\begin{itemize}
	\item de grootte van de populatie van klanten
	\item het type aankomstproces 
	\item de capaciteit van de wachtrij
	\item het type serviceproces
	\item het aantal beschikbare servers
\end{itemize}

\subsection{Grootte van de klantenpopulatie}
Er wordt een onderscheidt gemaakt tussen eindige en oneindige klantenpopulaties. Om gebruik te kunnen maken van de Partagoauto's moet je coöperant zijn van Partago. Het systeem kent dus ten alle tijden een bovengrens voor de klantenpopulatie en is dus eindig, namelijk het aantal coöperanten.

\subsection{Aankomstproces}
Elke Partago-coöperant heeft zijn eigen mobiliteitsbehoeften. We gaan er vanuit dat deze mobiliteitsbehoeften willekeurig en geheugenloos zijn. Geheugenloos wil zeggen dat het tijdstip van het huidige gebruik van het systeem onafhankelijk is van het vorige gebruik. Als er tijdens een bepaalde periode  $P_{1}$ een aantal reservaties waren dan heeft dit geen invloed op hoeveel reservaties er zullen zijn in een andere periode $P_{2}$. We willen het aankomstproces modeleren door middel van een statistische verdeling. Een discreet aankomstproces dat geheugenloos en willekeurig is kan gemodelleerd worden aan de hand van een Poisson-proces. Een Poisson-proces is een proces waarbij het aantal aankomsten verdeeld is volgens een Poisson-distributie terwijl de tijdens tussen 2 aankomsten een exponentiële verdeling volgen.

\subsection{Capaciteit van de wachtrij}
Wanneer een gebruiker niet geserveerd kan worden door Partago zal deze gebruiker zijn mobiliteitsbehoeften elders moeten vervullen. Er staan dus nooit gebruikers in de wachtrij. De grootte van de wachtrij is met andere woorden nul.

\subsection{Serviceproces}
Het Partago systeem serveert een gebruiker wanneer deze gebruiker gebruik maakt van de gereserveerde auto. De service tijden van alle gebruikers en servers zijn onafhankelijk van elkaar en worden veronderstelt exponentiëel verdeelt te zijn. Omdat er veronderstelt wordt dat de service tijden exponentieel verdeelt zijn gaan we er van uit dat deze ook geheugenloos zijn. Met andere woorden: de kans dat een gebruiker zijn gebruik van de auto beëindigt is onafhankelijk van de tijd reeds gebruik gemaakt van de auto. Dit is in de praktijk echter niet waar voor elke gebruiker. Bijvoorbeeld: hoe langer het gebruik, hoe meer de gebruiker betaalt. Dit kan een intrinsieke motivatie zijn om het gebruik sneller te beëindigen.

\subsection{Beschikbare servers}
Het systeem bestaat uit een vast aantal auto's. Het systeem weet ook ten alle tijden welke auto's beschikbaar zijn en welke niet.


\section{Constraint Satisfaction Problem}

\subsection{Definitie}
Het toewijzen welke gebruiker welke auto krijgt kan herleidt worden tot een Constraint Satisfaction Problem in het kort een CSP. Een CSP is gedefinieerd door een set \textbf{variabelen} $X_{1}$, $X_{2},...,X_{n}$ en een set \textbf{constraints} $C_{1}, C_{2},...C_{m}$. Elke variabele $X_{i}$ heeft een niet leeg \textbf{domein} $D_{i}$ van mogelijke waarden. Elke constraint $C_{i}$ heeft betrekking tot een subset van de variabelen en specifieert de mogelijke combinaties van waarden voor deze subset. Een toestand van het probleem is gedefinieerd als een \textbf{toewijzing} van waarden aan sommige van de variabelen $X_{i} =  v_{i}, X_{j} = v_{j}$... Een toewijzing dat geen enkele van de constraints schendt is een \textbf{consistente} of legale toewijzing. Een volledige toewijzing of \textbf{oplossing} van het CSP is een toewijzing waarin alle variabelen vernoemd worden en aan alle constraints is voldaan. Soms moet een oplossing van een CSP er ook naar streven een strafscore te minimaliseren \autocite{norvig}.

\subsection{Oplossen van een CSP}