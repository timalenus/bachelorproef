%%=============================================================================
%% Voorwoord
%%=============================================================================

\chapter*{\IfLanguageName{dutch}{Woord vooraf}{Preface}}
\label{ch:voorwoord}

%% TODO:
%% Het voorwoord is het enige deel van de bachelorproef waar je vanuit je
%% eigen standpunt (``ik-vorm'') mag schrijven. Je kan hier bv. motiveren
%% waarom jij het onderwerp wil bespreken.
%% Vergeet ook niet te bedanken wie je geholpen/gesteund/... heeft
Deze bachelorproef is het sluitstuk van mijn opleiding toegepaste informatica en één van de vele uitdagende projecten die ik binnen het bedrijf Partago heb kunnen verwezenlijken. Ik heb de kans gekregen tijdens het tweede semester van dit academiejaar mijn stage te volgen bij Partago. Na mijn stage ben ik ook tijdelijk aangenomen als ontwikkelaar bij Partago. Ik heb me de voorbije maanden dus zeer goed kunnen verdiepen in het Partago platform en hoe dit systeem voor het delen van elektrische auto's werkt. Met deze bachelorproef wil ik mee helpen vooruit denken hoe het Partago platform verder kan evolueren naarmate de gebruikersaantallen toenemen. De eerste ideeën voor het onderzoek beschreven in deze tekst kwamen voort uit een probleemstelling van mijn co-promotor Rik. Tijdens het eerste semester volgde ik het vak Artificiële Intelligentie en met dit onderzoek ging ik meer praktisch aan de slag met enkele concepten uit dat onderzoeksdomein.

Ik hoop de lezer van deze bachelorproef te kunnen boeien door het toepassen van een innoverend onderzoeksdomein, artificiële intelligentie, op een probleemstelling uit een innoverend sector: elektrisch autodelen. Ik hoop dat de uitkomst van het onderzoek ideeën doet opwekken over hoe een reservatiesysteem, en specifiek het reservatiesysteem van Partago, verder kan evolueren.  
 
Het uitvoeren van dit onderzoek en het schrijven van deze onderzoekstekst was een grote, maar voldoening gevende uitdaging. Mijn bachelorproef was echter nooit tot stand gekomen zonder de input en feedback van mijn promotor Wim Goedertier en co-promotor Rik Bellens. Mede dankzij hen kan ik deze kwalitatieve bachelorproef voorleggen. Tenslotte wil ik dit voorwoord afsluiten met het bedanken van mijn vriendin Liselotte, voor de steun en toeverlaat tijdens de vele moeilijke momenten die een uitdagend studie- en werkleven met zich meebrengen.

