%==============================================================================
% Sjabloon onderzoeksvoorstel bachelorproef
%==============================================================================
% Gebaseerd op LaTeX-sjabloon ‘Stylish Article’ (zie voorstel.cls)
% Auteur: Jens Buysse, Bert Van Vreckem

\documentclass[fleqn,10pt]{voorstel}

%------------------------------------------------------------------------------
% Metadata over het voorstel
%------------------------------------------------------------------------------

\JournalInfo{HoGent Bedrijf en Organisatie}
\Archive{Bachelorproef 2018 - 2019} % Of: Onderzoekstechnieken

%---------- Titel & auteur ----------------------------------------------------

% TODO: geef werktitel van je eigen voorstel op
\PaperTitle{Opschalen van een autodeelsysteem: optimalisatie en de hieruit vloeiende architectuuraanpassingen}
\PaperType{Onderzoeksvoorstel Bachelorproef} % Type document

% TODO: vul je eigen naam in als auteur, geef ook je emailadres mee!
\Authors{Tim Alenus\textsuperscript{1}} % Authors
\CoPromotor{Rik Bellens\textsuperscript{2} (Partago)}
\affiliation{\textbf{Contact:}
  \textsuperscript{1} \href{mailto:tim.alenus.y9731@student.hogent.be}{tim.alenus.y9731@student.hogent.be};
  \textsuperscript{2} \href{mailto:rik@partago.be}{rik@partago.be};
}

%---------- Abstract ----------------------------------------------------------

\Abstract{Partago is een coöperatie die een betere leefomgeving wil creeëren. Dit doen ze door elektrische deelauto's aan hun coöperanten ter beschikking te stellen. Hiervoor heeft Partago een eigen webapplicatie met reserveringssysteem ontwikkeld. Nu Partago groeit en er meer auto's en meer coöperanten zijn voldoet het huidige reserveringssysteem niet meer aan de noden om het complexer reserveringsvraagstuk efficiënt op te lossen. Het te voeren onderzoek is belangrijk om het groeiend aantal coöperanten blijvend tevreden te stellen. Door het gebruik van de deelauto's te optimaliseren kunnen er met minder deelauto's toch meer coöperanten bediend worden. Het te voeren onderzoek zal uit drie delen bestaan. Eerst onderzoeken we hoe we het reserveringsvraagstuk optimaler kunnen oplossen door gebruik te maken van een \textit{planning engine} die met behulp van artificiële intelligentie betere beslissingen zal kunnen maken. Deze \textit{planning engine} moet geïntegreerd worden in de huidige architectuur van het systeem en huidige reserveringen moeten gemigreerd kunnen worden. Hoe dit dient te gebeuren is het tweede luik van het onderzoek. Als laatste luik van de scriptie zullen de gevolgen voor de \textit{user-interface} en gebruikerservaring geïdentificeerd worden. Als Partago meer en meer coöperanten ten dienste wil zijn is het opschalen van het reservatiesysteem een kritische stap. De beoogde bachelorproef wil het plan uittekenen om deze kritische stap op een doordachte manier te nemen. Het resultaat van de bachelorproef zal als leidraad dienen voor de ontwikkelaars van Partago om het geoptimaliseerde reservatiesysteem te implementeren. In dit document wordt beschreven waar het huidige reservatiesysteem juist tekortschiet en hoe artificiële intelligentie een oplossing kan bieden om reservaties te optimaliseren. De drie stappen in het te voeren onderzoek wordt verder uitgelicht en er wordt een korte voorspelling gedaan van de verwachte resultaten en conclusies.  
}

%---------- Onderzoeksdomein en sleutelwoorden --------------------------------
% TODO: Sleutelwoorden:
%
% Het eerste sleutelwoord beschrijft het onderzoeksdomein. Je kan kiezen uit
% deze lijst:
%
% - Mobiele applicatieontwikkeling
% - Webapplicatieontwikkeling
% - Applicatieontwikkeling (andere)
% - Systeembeheer
% - Netwerkbeheer
% - Mainframe
% - E-business
% - Databanken en big data
% - Machineleertechnieken en kunstmatige intelligentie
% - Andere (specifieer)
%
% De andere sleutelwoorden zijn vrij te kiezen

\Keywords{Webapplicatieontwikkeling --- Machineleertechnieken en kunstmatige intelligentie  --- Opschalen --- Operationeel Onderzoek} % Keywords
\newcommand{\keywordname}{Sleutelwoorden} % Defines the keywords heading name

%---------- Titel, inhoud -----------------------------------------------------

\begin{document}

\flushbottom % Makes all text pages the same height
\maketitle % Print the title and abstract box
\tableofcontents % Print the contents section
\thispagestyle{empty} % Removes page numbering from the first page

%------------------------------------------------------------------------------
% Hoofdtekst
%------------------------------------------------------------------------------

% De hoofdtekst van het voorstel zit in een apart bestand, zodat het makkelijk
% kan opgenomen worden in de bijlagen van de bachelorproef zelf.
%---------- Inleiding ---------------------------------------------------------
\section{Introductie} % The \section*{} command stops section numbering
\label{sec:introductie}
Partago is een coöperatie (co-op) met als missie een leefbare omgeving te helpen creëren. Dit doen we door samen elektrisch te rijden. Partago stelt elektrische deelauto's ter beschikking aan zijn leden. Hiervoor heeft Partago een eigen platform ontwikkeld. In 2019 wil Partago als co-op met IT-capaciteit, via een overkoepelende organisatie, het deelplatform ter beschikking stellen aan andere elektrische deelautocoöperaties over heel Europa.

Partago is oorspronkelijk ontstaan in Gent, maar de interesse vanuit andere gemeenten neemt volop toe. Ook in Gent zelf groeit de coöperatie. De vloot in eigen beheer neemt dan ook aan een sneltempo toe. 

Bij het opschalen van een systeem denkt men klassiek eerst aan de niet-functionele vereisten. De groeiende vloot en het groeiend aantal gebruikers botst echter ook op enkele functionele limitaties van het huidige systeem. Het huidige reservatiesysteem dient opgeschaald te worden om met de meer complexe omgang van de coöperatie en die van andere coöperaties te kunnen omgaan. In wat volgt zullen de huidige functionele limitaties geschetst worden. 

%---------- Stand van zaken ---------------------------------------------------

\section{Stand van zaken}
\label{sec:stand-van-zaken}
\subsection{Huidig reservatiesysteem}
Vooraleer coöperanten gebruik mogen maken van een deelauto dient deze gereserveerd worden. Auto's staan verspreid over het werkingsgebied. Werkingsgebieden zijn onderverdeeld in verschillende zones of wijken. Een zone is de thuisbasis van een auto. Binnen één zone kunnen evenwel verschillende parkeerplaatsen zijn waar de auto zich bevindt: een publieke laadpaal, een parkeerplaats gereserveerd voor deelauto's, etc. Via de Partago app kunnen er reservaties worden gemaakt voor een specifieke auto. Indien de auto beschikbaar is wordt deze toegewezen aan het account van de gebruiker en kan de gebruiker de auto openen met de app. Tussen twee reservaties wordt een buffertijd voorzien zodat de auto's terug kunnen opladen.   

\subsection{Basisprobleem}
Gebruikers maken een reservatie per auto. Nu Partago groeit geeft dit echter geen optimaal gebruik van auto's wanneer er in één zone meerdere auto's zijn. 
We schetsen het probleem dat zich voordoet aan de hand van een eenvoudig voorbeeld: 

In de stationswijk zijn er 2 Partago deelauto's. Jan maakt een reservatie voor auto1 van 10h-12h. Piet maakt een reservatie voor auto2 van 14h-16h. Ilse, die ook woont in de stationswijk, wil een auto reserveren tussen 11h-15h. Het systeem zal echter melden dat dit niet mogelijk is. Om 11h is auto1 nog in gebruik en om 15h is auto2 al bezet. Moesten echter de reservaties van Jan en Piet beide voor auto1 zijn kon Ilse wel auto2 gebruiken.

Bovenstaand eenvoudig voorbeeld wordt snel heel complex wanneer er meerdere gebruikers, reservatieaanvragen en deelauto's bijkomen in het vraagstuk.
Een planningprobleem behoord tot de NP-complete problemen \autocite{negnevitsky}. Zulke problemen 

\subsection{Bijkomende complexiteiten}


% Voor literatuurverwijzingen zijn er twee belangrijke commando's:
% \autocite{KEY} => (Auteur, jaartal) Gebruik dit als de naam van de auteur
%   geen onderdeel is van de zin.
% \textcite{KEY} => Auteur (jaartal)  Gebruik dit als de auteursnaam wel een
%   functie heeft in de zin (bv. ``Uit onderzoek door Doll & Hill (1954) bleek
%   ...'')

Je mag gerust gebruik maken van subsecties in dit onderdeel.

%---------- Methodologie ------------------------------------------------------
\section{Methodologie}
\label{sec:methodologie}

Hier beschrijf je hoe je van plan bent het onderzoek te voeren. Welke onderzoekstechniek ga je toepassen om elk van je onderzoeksvragen te beantwoorden? Gebruik je hiervoor experimenten, vragenlijsten, simulaties? Je beschrijft ook al welke tools je denkt hiervoor te gebruiken of te ontwikkelen.

%---------- Verwachte resultaten ----------------------------------------------
\section{Verwachte resultaten}
\label{sec:verwachte_resultaten}



Hier beschrijf je welke resultaten je verwacht. Als je metingen en simulaties uitvoert, kan je hier al mock-ups maken van de grafieken samen met de verwachte conclusies. Benoem zeker al je assen en de stukken van de grafiek die je gaat gebruiken. Dit zorgt ervoor dat je concreet weet hoe je je data gaat moeten structureren.

%---------- Verwachte conclusies ----------------------------------------------
\section{Verwachte conclusies}
\label{sec:verwachte_conclusies}

Hier beschrijf je wat je verwacht uit je onderzoek, met de motivatie waarom. Het is \textbf{niet} erg indien uit je onderzoek andere resultaten en conclusies vloeien dan dat je hier beschrijft: het is dan juist interessant om te onderzoeken waarom jouw hypothesen niet overeenkomen met de resultaten.



%------------------------------------------------------------------------------
% Referentielijst
%------------------------------------------------------------------------------
% TODO: de gerefereerde werken moeten in BibTeX-bestand ``voorstel.bib''
% voorkomen. Gebruik JabRef om je bibliografie bij te houden en vergeet niet
% om compatibiliteit met Biber/BibLaTeX aan te zetten (File > Switch to
% BibLaTeX mode)

\phantomsection
\printbibliography[heading=bibintoc]

\end{document}
